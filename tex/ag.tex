We have previously seen in Sections~\ref{sec:theo_permutation_groups}
and~\ref{sec:theo_partial_permutation_inverse_monoids} how permutation groups
and partial permutation inverse monoids can describe symmetries and partial
symmetries of undirected graphs respectively. This implies that we can
capture the (partial) symmetries inherent in abstract MPSoC architectures by first
modelling them as undirected graphs and then determining the automorphism
groups and the partial automorphism inverse monoids of these graphs, which
we shall term \textit{architecture graphs}.

In this chapter we will first formally define architecture graphs in
Section\ref{sec:ag_definition}. In
Section\ref{sec:ag_determining_automorphisms} we will then examine how we can
determine a generating set for the automorphism group of an architecture
graph and in Section\ref{sec:ag_automorphism_decomposition} we demonstrate how
we can decompose the automorphism groups of certain \textit{separable} and
\textit{hierarchical} architecture graphs into direct and wreath products
respectively. Finally, in
Section\ref{sec:ag_determining_partial_automorphisms} we describe a simple but
inefficient algorithm for finding a generating set for the partial automorphism
inverse monoid of an architecture graph.

\section{Definition}
\label{sec:ag_definition}

Transforming abstract MPSoC architectures into undirected graphs is easy: We
introduce a vertex for every processing element and an edge for every
communication channel. It should now be apparent why we have only considered
undirected graphs so far: MPSoC communication channels are usually
bi-directional\footnote{But note that this need not be true in general, we can
imagine e.g. one-directional FIFO channels between processing elements or
shared memory that can be read by multiple processing elements but only be
written to by a subset of them.  Nevertheless we will continue assuming that
all communication channels are bi-directional to keep notation simple. The
following discussion is trivially extensible to directed architecture graphs.}.
In addition, we need to take into account the fact that processing elements and
communication channels might be heterogeneous. Thus we introduce
\textit{vertex} and \textit{edge colors} which assign identical processing
elements and communication channels the same unique color (where we simply
represent colors by positive integers). More formally we define an
architecture graph as follows:

\begin{defn}{Architecture Graph}
  An architecture graph $A$ is a four-tuple $(P, C, \operatorname{col}_P,
  \operatorname{col}_C)$ where:
  %
  \begin{itemize}
    \item $P$ is a set of vertex indices of the form $\{1, 2, \dots, n \mid n
          \in \mathbb{N}_+\}$.
    \item $C = \{\{i, j\} \mid i \leq j \land$ an edge exists
                 between $i$ and $j$, $i, j \in P\}$\footnote{Note that $C$ is a
                 multiset.}.
    \item $\operatorname{col}_P: P \rightarrow \mathbb{N}_+$ maps every $p \in
          P$ to some color.
    \item $\operatorname{col}_C: C \rightarrow \mathbb{N}_+$ maps every $c \in C$
          to some color.
  \end{itemize}
\end{defn}

\begin{exmp}[label=exmp:arch_graph]{Architecture Graph}
  Consider the abstract MPSoC architecture presented in
  Figure~\ref{fig:tcol_simple_arch}. Take note of the fact that unlike
  before we are now dealing with an MPSoC architecture with two types of
  processing elements and communication channels respectively.  The different
  processor types are indicated by different shades of gray and the communication
  channels arise from L1 caches shared between processing elements of the same
  type and an L2 cache shared by all processing elements. If we omit any loops
  for the sake of simplicity, the corresponding architecture graph can described
  as follows, see Figure~\ref{fig:tcol_simple_arch_graph} for a graphical
  representation:
  %
  \begin{itemize}
    \item $P = \{1,2,3,4\}$
    \item $C = \{\{1,2\},\{1,2\},\{1,3\},\{1,4\},\{2,3\},\{2,4\},\{3,4\},\{3,4\}\}$
    \item $\operatorname{col}_P(1) = 1,\ \operatorname{col}_P(2) = 1,
           \operatorname{col}_P(3) = 2,\ \operatorname{col}_P(4) = 2$
    \item $\operatorname{col}_C(\{1,2\}) = 1,\ \operatorname{col}_C(\{3,4\}) = 1,\ 
           \operatorname{col}_C(\{1,2\}) = 2,\ \operatorname{col}_C(\{1,3\}) = 2,\\
           \operatorname{col}_C(\{1,4\}) = 2,\ \operatorname{col}_C(\{2,3\}) = 2,\ 
           \operatorname{col}_C(\{2,4\}) = 2,\ \operatorname{col}_C(\{3,4\}) = 2$
  \end{itemize}
  %
  \begin{figure}[H]
    \centering
    \begin{subfigure}{.55\textwidth}
      \centering
      \includeressource[width=.8\textwidth]{simple_arch.pdf}
      \caption{}
      \label{fig:tcol_simple_arch}
    \end{subfigure}
    %
    \hspace{1cm}
    %
    \begin{subfigure}{.3\textwidth}
      \centering
      \begin{tikzpicture}
        % vertices
        \Vertex[x=0,y=0,L=$1$]{1}
        \Vertex[x=2,y=0,L=$2$]{2}
        {\renewcommand{\VertexLightFillColor}{LightGray}
         \Vertex[x=2,y=-2,L=$3$]{3}
         \Vertex[x=0,y=-2,L=$4$]{4}}
        % L2 edges
        \Edge[label=L2,labelcolor=tcolorboxGray](1)(2)
        \Edge[label=L2,labelcolor=tcolorboxGray](1)(3)
        \Edge[label=L2,labelcolor=tcolorboxGray](1)(4)
        \Edge[label=L2,labelcolor=tcolorboxGray](2)(3)
        \Edge[label=L2,labelcolor=tcolorboxGray](2)(4)
        \Edge[label=L2,labelcolor=tcolorboxGray](3)(4)
        % L1 edges
        \tikzstyle{EdgeStyle}=[bend angle=45, bend left]
        \Edge[label=L1,labelcolor=tcolorboxGray](1)(2)
        \Edge[label=L1,labelcolor=tcolorboxGray](3)(4)
      \end{tikzpicture}
      \caption{}
      \label{fig:tcol_simple_arch_graph}
    \end{subfigure}
    \caption{From Abstract MPSoC architecture to automorphism graph.}
  \end{figure}
\end{exmp}

\section{Determining Automorphisms}
\label{sec:ag_determining_automorphisms}

Finding a generating set for the automorphism group of a graph is a
well-studied problem in graph theory for which there exist efficient
algorithms. The program \textit{nauty/Traces}~\cite{nauty} is able to determine
such generating sets for vertex colored graphs. Since automorphism graphs can
be totally colored we furthermore need to be able to transform any
totally colored graph into a vertex colored graph with the same automorphism
group, which is possible via the following therom:

\begin{thm}{Reducing Totally Colored Architecture Graphs}
  Given a totally colored architecture graph $A = (P, C, \operatorname{col}_P,
  \operatorname{col}_C)$ and letting
  %
  \begin{align*}
    |\operatorname{col}_P| &= \text{the number of distinct vertex colors of A} \\
    |\operatorname{col}_C| &= \text{the number of distinct edge colors of A}
  \end{align*}
  %
  and
  %
  \begin{align*}
    \operatorname{p}&: P' \rightarrow P,\ p \mapsto ((p - 1) \bmod |P|) + 1 \\
    \operatorname{lvl}&: P' \rightarrow P,\ p \mapsto \floor*{\frac{p - 1}{|P|}} \\
    \operatorname{bd}_i&: \mathbb{N}_+ \rightarrow \{0,1\},\
    n \mapsto \text{the ith digit of the binary representation of } n
  \end{align*}
  %
  we construct a vertex colored architecture graph $A' = (P', C',
  \operatorname{col}_P', \operatorname{col}_C')$ as follows:
  %
  \begin{itemize}
    \item $P' = \{1, 2, \dots, |P| \cdot (\floor*{\log_2(|\operatorname{col}_C|)} + 1)\}$
    \item
      $\begin{aligned}
        C' =\ &\{\{i, j\} \mid j = i + |P|,\ i, j \in P'\} \\
        \cup\ &\{\{i, j\} \mid
          l = \operatorname{lvl}(i) = \operatorname{lvl}(j)
            \land
          \{\operatorname{p}(i), \operatorname{p}(j)\} \in C
            \land
          \operatorname{bd}_l(\operatorname{col}_C(\{i,j\})) = 1,\ i, j \in P'\}
      \end{aligned}$
    \item $\forall p \in P': \operatorname{col}_P'(p) =
      \operatorname{col}_P(p) + |\operatorname{col}_P| \cdot \operatorname{lvl}(p)$
    \item $\forall c \in C': \operatorname{col}_C'(c) = 1$
  \end{itemize}
  %
  $A$ and $A'$ then have the same automorphism group, a proof of which is beyond
  the scope of this thesis.
\end{thm}

\begin{exmp}{Reducing Totally Colored Architecture Graphs}
  Consider again the totally colored architecture graph presented in
  Example~\ref{exmp:arch_graph}. A vertex colored architecture graph with
  the same automorphism group can be described as follows, see
  Figure~\ref{fig:simple_arch_graph_vcol} for a graphical representation:
  %
  \begin{itemize}
    \item $P' = \{1,2,3,4,5,6,7,8\}$
    \item $C' = \{\{1,2\},\{3,4\},\{1,5\},\{2,6\},\{3,7\},\{4,8\},\{5,6\},
                  \{5,7\},\{5,8\},\{6,7\},\{6,8\},\{7,8\}\}$
    \item $\operatorname{col}'_P(1) = 1,\ \operatorname{col}'_P(2) = 1,\ 
           \operatorname{col}'_P(3) = 2,\ \operatorname{col}'_P(4) = 2,\\
           \operatorname{col}'_P(5) = 3,\ \operatorname{col}'_P(6) = 3,\ 
           \operatorname{col}'_P(7) = 4,\ \operatorname{col}'_P(8) = 4$
    \item $\forall c \in C: \operatorname{col}'_C(c) = 1$
  \end{itemize}
  %
  \begin{figure}[H]
    \centering
    \begin{tikzpicture}
      % vertices
      \Vertex[x=0,y=-1.5,L=$1$]{1}
      \Vertex[x=1.5,y=-1.5,L=$2$]{2}
      {\renewcommand{\VertexLightFillColor}{LightGray}
        \Vertex[x=3,y=-1.5,L=$3$]{3}
        \Vertex[x=4.5,y=-1.5,L=$4$]{4}}
      {\renewcommand{\VertexLightFillColor}{MediumGray}
        \Vertex[x=0,y=0, L=$5$]{5}
        \Vertex[x=1.5,y=0, L=$6$]{6}}
      {\renewcommand{\VertexLightFillColor}{DarkGray}
        \Vertex[x=3,y=0, L=$7$]{7}
        \Vertex[x=4.5,y=0, L=$8$]{8}}
      % straight edges
      \Edge(1)(5)
      \Edge(2)(6)
      \Edge(3)(7)
      \Edge(4)(8)
      % bent edges
      \tikzstyle{EdgeStyle}=[bend right]
      \Edge(1)(2)
      \Edge(3)(4)
      %
      \tikzstyle{EdgeStyle}=[bend left]
      \Edge(5)(6)
      \Edge(5)(7)
      \Edge(5)(8)
      \Edge(6)(7)
      \Edge(6)(8)
      \Edge(7)(8)
    \end{tikzpicture}
    \caption{Vertex colored automorphism graph.}
    \label{fig:simple_arch_graph_vcol}
  \end{figure}
\end{exmp}
%
Thus we can construct the automorphism group of any architecture graph by first
transforming the graph into an isomorphic vertex colored graph, finding a
generating set for its automorphism group using nauty and then a corresponding
BSGS using the algorithms presented in Chapter~\ref{chap:bsgs}.

\section{Automorphism Decomposition}
\label{sec:ag_automorphism_decomposition}

The architecture graphs we are dealing with in practice are based on real world
abstract MPSoC architectures. Many of these are either \textit{separable} or
\textit{hierarchical} in nature, properties that we, as it turns out, can use
to decompose the autmorphism groups of the corresponding architecture graphs
into direct and wreath products which we introduced in
Section\ref{sec:pg_direct_and_wreath_product}. This is not just an
interesting observation but can also help us solve the task mapping problem
more efficiently for these types of architecture graphs as discussed in
Section\ref{sec:tmor_optimization_via_decomposition}.

Separable architecture graphs are all those made up of ``islands'' of
processing elements for which no automorphism maps from one island to another.
This occurs e.g. for abstract MPSoC architectures with several distinct
processor types. Consider e.g. the \textit{Exynos} architecture shown in
Figure~\ref{fig:exynos}. Its automorphism group $G_{\mathrm{Exynos}}$
is the direct product of the automorphism groups of the subgraphs created by
restricting that architecture graph to one of the two processor types, i.e.:

\begin{align*}
  G_{\mathrm{Exynos}} &=
      \left<\{(1\ 2), (2\ 3), (2\ 4), (3\ 4)\}\right>
        \times
      \left<\{(5\ 6), (6\ 7), (6\ 8), (7\ 8)\}\right> \\
    &= \left<\{1\ 2), (2\ 3), (2\ 4), (3\ 4), (5\ 6), (6\ 7), (6\ 8), (7\ 8)\}\right>
\end{align*}
%
Hierarchical architecture graphs on the other hand arise from abstract
architecture graphs that are made up of interconnected identical ``clusters''
of processing elements. Consider e.g. the \textit{HAEC} architecture shown in
Figure~\ref{fig:haec}. We can model the corresponding architecture graph as a
hypergraph according to Definition~\ref{defn:hypergraph}. Its automorphism
group $G_{\mathrm{HAEC}}$ is then the wreath product of one clusters automorphism group and the
autormophism group of the ``cluster graph'' itself, i.e.:

\begin{align*}
  &\begin{aligned}
    \mathllap{G_{\mathrm{HAEC,proto}}} =
      \langle\{&(1, 4)(2, 3)(5, 8)(6, 7)(9, 12)(10, 11)(13, 16)(14, 15), \\
               &(2, 5)(3, 9)(4, 13)(7, 10)(8, 14)(12, 15)\}\rangle
    \end{aligned} \\
  &\mathllap{G_{\mathrm{HAEC,super}}} = \left<\{(1\ 4)(2\ 3)\}\right> \\
\end{align*}
%
and:
%
\begin{equation*}
  G_{\mathrm{HAEC}} = G_{\mathrm{HAEC,proto}} \wr G_{\mathrm{HAEC,super}}
\end{equation*}
%
We can either manually specify whether a given architecture graph is separable
or hierarchical or automatically detect this property by using the algorithms
presented in~\cite{Donaldson} which are able to deterministically discover
possible direct and wreath product decompositions for any given permutation
group. These algorithms are implemented in \texttt{mpsym} but their inner
working is beyond the scope of this thesis.

\section{Determining Partial Automorphisms}
\label{sec:ag_determining_partial_automorphisms}

Unfortunately, there seems to be no existing research on efficient algorithms
for determining a generating set for a graphs partial automorphism inverse
monoid.

Determining whether a given partial permutation is a partial automorphism of an
architecture graph is trivial. However, the total number of possible partial
permutations of $\{1, \dots, |P|\}$ is $\sum_{i=0}^{|P|} \frac{|P|!}{(|P| -
i)!} = \left(\sum_{i=0}^{|P|} \frac{1}{i!}\right) (|P| + 1)! \approx e \cdot
(|P| + 1)!$ and thus generally very large. To avoid searching through all of
them to find a suitable generating set we can make use of the fact that if a
partial permutation is not contained in a partial permutation inverse monoid
then neither are any of its \textit{extensions} (or their extensions and so
on), where we define extensions as follows:

\begin{defn}{Extensions of a Partial Permutation}
  Given a partial permutation $s$ of a set $\Omega$, an extension of $s$ is
  another partial permutation $t$ with $\operatorname{dom}(t) =
  \operatorname{dom}(s) \cup \{x \mid x \notin \operatorname{dom}(s), x \in
  \Omega\}$ and $\operatorname{im}(t) = \operatorname{im}(s) \cup \{x \mid x
  \notin \operatorname{im}(s), x \in \Omega\}$. We refer to all extensions of $s$
  (of which there are $|\operatorname{dom}(s)|^2$) by $\operatorname{ext}(s)$ and
  to the set resulting from a recursive application of $\operatorname{ext}$ as
  $\operatorname{ext}^*(s)$.
\end{defn}

\begin{exmp}{Extensions of a Partial Permutation}
  Let $\Omega = \Omega_4$ and $s = (1, 2)$, i.e. $\operatorname{dom}(s) =
  \operatorname{im}(s) = \{1,2\}$, then $\operatorname{ext}(s) = \{(1, 2)(3),
  (1, 2)(4), (1, 2)[3, 4], (1, 2)[4, 3]\}$.
\end{exmp}
%
So formally, it holds that if $s$ is not a partial automorphism of an
architecture graph then neither are any of the partial permutations in
$\operatorname{ext}^*(s)$.

Algorithm~\ref{alg:partial_autom_generators} is based on this. If we invoke
this recursive algorithm with $s = 0$ and $M$ initially only containing $0$ and
$1_{\{1,\dots,|P|\}}$, then the algorithm will traverse a search tree of all
possible partial permutations $s$ of $\{1, \dots, |P|\}$ in depth first
fashion and adjoin any $s \notin M$ that is a partial automorphism of the
architecture graph $A$ to $M$\footnote{This of course assumes that we can
efficiently modify our representation of $M$ in a suitable way such as to adjoin
$s$ to $M$, refer to~\cite{Mitchell} to see that this is indeed possible.}.
Because the nodes of any subtree of this search tree with root $s$ are exactly
$\operatorname{ext}^*(s)$, we can completely prune such a subtree if $s$ is not
a partial automorphism of $A$.

\begin{algorithm}
  \caption{Find partial automorphism inverse monoid generating set}
  \label{alg:partial_autom_generators}
  \begin{algorithmic}[1]
  \Procedure{PARTIAL\_AUTOM\_GENERATORS}{$A$, $s$, $M$}
    \If{$s$ is a partial automorphism of $A$}
      \State Adjoin $s$ to $M$
    \Else
      \State \textbf{return}
    \EndIf
    \\
    \For{Every extension $s'$ of $s$}
        \State PARTIAL\_AUTOM\_GENERATORS($A$, $s'$, $M$)
    \EndFor
  \EndProcedure
  \end{algorithmic}
\end{algorithm}

Unfortunately, preliminary experiments have shown that in practice,
Algorithm~\ref{alg:partial_autom_generators} is too slow for all but the
smallest architecture graphs. Developing a more efficient algorithm could be
the subject of further research.
