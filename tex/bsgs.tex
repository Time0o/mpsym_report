We will now discuss how to efficiently obtain a BSGS for any permutation group
$G$ given a generating set for $G$. This topic is vast and many variants of the
algorithms presented here, utilizing different optimizations and trade-offs, as
well as additional BSGS construction algorithms like the \textit{Solvable BSGS
Algorithm} have been published in the computational group theory literature.
Refer to e.g.~\cite{Holt}, Chapter 4 and 6 for a deeper treatment.

As a preliminary matter, in
Section~\ref{sec:bsgs_representing_basic_orbits_and_transversals} we will see
how to construct the basic orbits $\Delta^{(i)}$ and transversals $U^{(i)}$
from $\beta_i$ and $S^{(i)}$.
%
In Section~\ref{sec:bsgs_the_deterministic_schreier_sims_algorithm} we will
then present the first BSGS construction algorithm, the \textit{Deterministic
Schreier-Sims Algorithm}.
%
Finally, in Section~\ref{sec:bsgs_the_random_schreier_sims_algorithm} we
present an alternative Monte Carlo BSGS construction algorithm, the
\textit{Random Schreier-Sims Algorithm}.

\section{Representing Basic Orbits and Transversals}
\label{sec:bsgs_representing_basic_orbits_and_transversals}

Since basic orbits are simply unordered sets of elements of $\Omega$, we can
represent them using hash tables and then perform orbit membership tests
in $O(1)$ time (on average). We can obtain $\Delta^{(i)}$ from $\beta_i$ and
$S^{(i)}$ via Algorithm~\ref{alg:basic_orbit} which is a simple fixed-point
algorithm.

\begin{algorithm}
  \caption{Determine basic orbit.}
  \label{alg:basic_orbit}
  \begin{algorithmic}[1]
    \Procedure{BASIC\_ORBIT}{$\beta_i,S^{(i)}$}
       \State $\Delta^{(i)} \gets \{\beta_i\}$
       \\
       \While{$\Delta^{(i)}$ is changing}
         \For{$x \in \Delta^{(i)}$, $g \in S^{(i)}$}
            \State $\Delta^{(i)} \gets \Delta^{(i)} \cup \{g(x)\}$
         \EndFor
       \EndWhile
       \\
       \State \textbf{return} $\Delta^{(i)}$
    \EndProcedure
  \end{algorithmic}
\end{algorithm}

While it is possible to similarly store all $u_x^{(i)} \in U^{(i)}$ explicitly,
this can be impractical for groups of large degree because $|U^{(i)}| =
|\Delta^{(i)}|$ can be as large as $|\Omega|$. A possible workaround is storing
$U^{(i)}$ as a tree structure from which we can reconstruct the $u_x^{(i)}$ on
demand. These tree structures are commonly implemented as \textit{(Shallow)
Schreier Vectors}, refer to e.g.~\cite{Holt} Chapter 4.

Since for the purpose of this thesis we can assume that $|\Omega| = |\Omega_n|$
is of manageable size\footnote{Because MPSoCs generally do not contain thousands of
processing elements, let alone millions.}, we will not consider Schreier Vectors
in detail and instead simply present
Algorithm~\ref{alg:basic_orbit_and_transversal}, a modified version of
Algorithm~\ref{alg:basic_orbit} that determines $U^{(i)}$ in addition to
$\Delta^{(i)}$. The necessary modifications should be self-explanatory.

\begin{algorithm}
  \caption{Determine basic orbit and transversal.}
  \label{alg:basic_orbit_and_transversal}
  \begin{algorithmic}[1]
    \Procedure{BASIC\_ORBIT\_AND\_TRANSVERSAL}{$\beta_i,S^{(i)}$}
       \State $\Delta^{(i)} \gets \{\beta_i\}$
       \State $u_{\beta_i} \gets 1$
       \\
       \While{$\Delta^{(i)}$ is changing}
         \For{$x \in \Delta^{(i)}$, $g \in S^{(i)}$}
            \If{$g(x) \notin \Delta^{(i)}$}
              \State $\Delta^{(i)} \gets \Delta^{(i)} \cup \{g(x)\}$
              \State $u_{g(x)}^{(i)} \gets u_x^{(i)} g$
            \EndIf
         \EndFor
       \EndWhile
       \\
       \State $U^{(i)} \gets \{u_x^{(i)} \mid x \in \Delta^{(i)}\}$
       \\
       \State \textbf{return} $\Delta^{(i)}$, $U^{(i)}$
    \EndProcedure
  \end{algorithmic}
\end{algorithm}

\section{The Deterministic Schreier-Sims Algorithm}
\label{sec:bsgs_the_deterministic_schreier_sims_algorithm}

Let's now turn our attention to a general deterministic algorithm which returns
a BSGS for any permutation group given a generating set for that group. This
algorithm is commonly refered to as the \textit{Schreier-Sims Algorithm}. Here
we will refer to it as the \textit{Deterministic Schreier-Sims Algorithm} to
differentiate it from the \textit{Random Schreier-Sims Algorithm} introduced in
Section~\ref{sec:bsgs_the_random_schreier_sims_algorithm}.

\begin{algorithm}
  \caption{Strip}
  \label{alg:strip}
  \begin{algorithmic}[1]
    \Procedure{STRIP}{$g,(B, S, \Delta, U)$}
       \State $g' \gets g$
       \\
       \For{$i = 1, 2, \dots, k$}
          \If{$g'(\beta_i) \notin \Delta^{(i)}$}
            \State \textbf{return} $g$, $i$
          \EndIf
          \\
          \State Find $u_i \in U^{(i)}$ such that $u_i(\beta_i) = g'(\beta_i)$
          \\
          \State $g' \gets g' u_i^{-1}$
       \EndFor
       \\
       \State \textbf{return} $g$, $k + 1$
    \EndProcedure
  \end{algorithmic}
\end{algorithm}

The Deterministic Schreier-Sims Algorithm makes use of
Algorithm~\ref{alg:strip}, a slightly modified version of
Algorithm~\ref{alg:is_member} which returns both $g'$ and the loop index $i$.
Importantly, since in every iteration we ``strip'' $u_i$ from $g'$ by
multiplying it with $u_i^{-1}$, at the start of a given iteration $i$ we know
that the returned $g'$ is of the form $g' = u_k u_{k-1} \cdots u_i$ and thus
stabilizes $\beta_1, \dots, \beta_{i-1}$.

\begin{algorithm}
  \caption{The Deterministic Schreier-Sims Algorithm}
  \label{alg:schreier_sims_deterministic}
  \begin{algorithmic}[1]
  \Procedure{SCHREIER\_SIMS\_DETERMINISTIC}{$B, S$}
    \LineComment{$B$ initially is an empty base.}
    \LineComment{$S$ initially is any generating set for $G$.}
    \\
    \State Append elements $x \in \Omega$ to $B$ until no $g \in G$ stabilizes $B$
    \\
    \For{$1 \leq i \leq k$}
      \State $S^{(i)} = S \cap G_{\beta_1, \beta_2, \dots, \beta_{i-1}}$
      \State $\Delta^{(i)}, U^{(i)} = \text{BASIC\_ORBIT\_AND\_TRANSVERSAL}(
        \beta_i, S^{(i)})$
    \EndFor
    \\
    \State{$i \gets 1$}
    \While{$i \geq 1$}
\State \unskip $\mathrm{next:}$
      \For{every generator $g$ of $G^{(i)}_{\beta_i}$}
        \State $g', j \gets \operatorname{STRIP}(g)$
        \\
        \If{$g' = 1$}
          \State \textbf{continue}
        \EndIf
        \\
        \If{$j = k + 1$}
          \State Append an $x \in \Omega$ with $g'(x) \ne x$ to $B$
          \State $S^{(k)} \gets \{\}$
        \EndIf
        \\
        \For{$l = i + 1, \dots, j$}
          \State $S^{(l)} \gets S^{(l)} \cup \{g'\}$
          \State $\Delta^{(l)}, U^{(l)} = \text{BASIC\_ORBIT\_AND\_TRANSVERSAL}(
            \beta_l, S^{(l)})$
        \EndFor
        \\
        \State $i \gets j$
        \State \textbf{goto} $\mathrm{next}$
      \EndFor
      \\
      \State $i \gets i - 1$
    \EndWhile
    \\
    \State \textbf{return} $(B,S,\Delta,U)$
  \EndProcedure
  \end{algorithmic}
\end{algorithm}

Algorithm~\ref{alg:schreier_sims_deterministic} is a high-level
description of the Deterministic Schreier-Sims Algorithm. It is based on a
simple observation: Referring back to Definition~\ref{defn:bsgs} it is easy to
see that for a valid BSGS with base of length $k$ for some permutation group
$G$ it holds that $G^{(k+1)} = 1$ and $G^{(i)}_{\beta_i} = G^{(i+1)}$ because
both $G^{(i)}$ and $G^{(i+1)}$ stabilize $\beta_1, \beta_2, \dots, \beta_{i-1}$
and $G^{(i+1)}$ additionally stabilizes $\beta_i$.

The Deterministic Schreier-Sims algorithm exploits this property as follows:
%
We start by extending an initially empty base $B$ until no $g \in G$ stabilizes
every $\beta \in B$, such that we have $G^{(k+1)} = 1$ (line 3) and then
initialize the basic stabilizer generating sets $S^{(i)}$ as well as the basic
orbits $\Delta^{(i)}$ and transversals $U^{(i)}$ (lines 5--8).

We then assure that $G^{(i)}_{\beta_i} = G^{(i + 1)}$ for $i = k, k-1, \dots,
1$. It is easy to see that $G^{(i + 1)} \leq G^{(i)}_{\beta_i}$ always holds
and so the problem reduces to assuring that $G^{(i)}_{\beta_i} \leq G^{(i +
1)}$.
%
Assuming for a moment that we know a generating set for $G^{(i)}_{\beta_i}$, we
can accomplish this by iterating over this generating set and adjoining any
generator $g$ that is not already an element of $G^{(i + 1)}$ to $S^{(i + 1)}$.
In fact, given $g', j = \mathrm{STRIP}(g)$ it suffices to adjoin $g'$ because
we can represent $g$ as $g$ = $g' u_{j-1} \cdots u_1$ where $u_{j-1}, \dots,
u_1$ are already elements of $U_{j-1}, \dots, U_1$ (lines 14--18).
%
Finding the generators of $G^{(i)}_{\beta_i}$ is possible by making use of
\textit{Schreier's Subgroup Lemma}, refer to~\cite{Holt} Chapter 4.

This process is complicated by the fact that in a certain iteration $i$ we need
to adjoin $g'$ not only to $S^{(i + 1)}$ but to $S^{(i + 1)}, \dots, S^{(j)}$
(lines 25--28) and then reset $i$ to $j$ because $G^{(j)}_{\beta_j} \leq G^{(j +
1)}$ might not hold anymore (lines 29--30).
%
If $j = k + 1$ then $g'$ stabilizes all $\beta \in B$ and thus we also need to
adjoin an additional base point not stabilized by $g'$ to $B$ such that $G^{(k
+ 1)} = 1$ remains true (lines 20--23).

For a permutation group $G$ acting on $\Omega_n$, the time complexity of the
Deterministic Schreier-Sims is polynomial in $n$. For an overview of the time
complexity of several variants of this algorithm refer to~\cite{Butler}
Chapter 14.

\section{The Random Schreier-Sims Algorithm}
\label{sec:bsgs_the_random_schreier_sims_algorithm}

While the Deterministic Schreier-Sims Algorithm is guaranteed to terminate and
produce a correct BSGS for every possible generating set it can be slow in
practice. With the Random Schreier-Sims Algorithm there exists a very fast
Monte Carlo algorithm with which we can potentially find a BSGS quickly,
even for groups of large degree, the drawback being that this BSGS is not
guaranteed to be valid.

Despite its name, the Random Schreier-Sims algorithm is not based on Schreier's
Subgroup Lemma at all but rather on the following observations:

\begin{itemize}
  \item It is possible to efficiently and uniformly sample elements from a
        permutation group without the need to construct a BSGS first, refer
        to~\cite{Holt} Chapter 3.

  \item Given some $g \in G$, it can be shown that Algorithm~\ref{alg:strip}
        returns a permutation $g' = 1$ with probability
        $< 1/2$ if $(B,S)$ is not a valid BSGS for $G$.
\end{itemize}

\noindent
Thus, given a BSGS $(B,S)$, if we randomly generate some $g \in G$ $w \in
\mathbb{N}_+$ times and $\mathrm{STRIP}(g, (B,S))$ returns the identity for all
of them, we know that $(B,S)$ is a BSGS for $G$ with probability $p \ge 1 -
2^{-w}$.

The Random Schreier-Sims Algorithm starts by initializing a BSGS as in
Algorithm~\ref{alg:schreier_sims_deterministic} (lines 3--8).  We then keep
generating random $g \in G$ and determine $g', j = \mathrm{STRIP}(g, (B,S))$.
If $g' \ne 1$ we extend our BSGS as in
Algorithm~\ref{alg:schreier_sims_deterministic}  (lines 20--28, with $i$ always
equal to 1). If $g' = 1$ $w$ consecutive times, we stop.  In certain cases we
also know $|G|$ ahead of time, e.g. when $G$ is a wreath product, such that we
can keep extending the BSGS until $|U^{(k)}| \cdot |U^{(k-1)}| \cdots |U^{(1)}|
= |G|$.

For reasonably large $w$ we can be very sure that the resulting BSGS is a valid
BSGS for $G$. If we want to be \textit{absolutely} sure that this is the case
we can:

\begin{itemize}
  \item Run the Deterministic Schreier-Sims Algorithm on the resulting BSGS.
  \item Use the Todd-Coxeter-Schreier-Sims or the Sims ``Verify'' Algorithm
        to correct the resulting BSGS. While this can be much faster for groups of
        large degree, these algorithms are beyond the scope of this
        thesis, refer to~\cite{Holt} Chapter 6.
\end{itemize}

\noindent
For a complete pseudocode description of the Random Schreier-Sims Algorithm refer
to e.g.~\cite{Holt} Chapter 4.
