An important problem pertaining to MPSoCs is that of intelligently mapping
(i.e.  assigning) computational tasks to processing elements, often at runtime.

Whether some mapping is comparatively better than another usually depends on
application specific optimality criteria that can only be evaluated by use of
computationally expensive simulation. Thus there is a need to reduce the
potentially very large mapping search space.

This thesis concerns itself with what we refer to as \textit{symmetry
reduction}, a technique that achieves this search space reduction by exploiting
symmetries inherent in many MPSoC architecutures and which can be used in
conjunction with existing methods that heuristically traverse the search space.

For this purpose we make use of well-established, as well as some recent state
of the art methods originated by researchers in the field of
\textit{computational group theory (CGT)}, a branch of computational
mathematics that concerns itself with the representation and analysis of many
of the algebraic structures underlying group theory.
%
We evaluate symmetry reduction results achieved by use of
\texttt{mpsym}~\cite{mpsym}, a framework written from the ground up
specifically to address symmetry reduction.
%
We demonstrate that \texttt{mpsym} is fast, especially for architectures that
are hierarchical in nature, and that it outperforms comparable general purpose
CGT frameworks like GAP~\cite{GAP4}.
