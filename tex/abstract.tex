This thesis concerns itself with what we refer to as \textit{symmetry
reduction} of multi-processor systems-on-chips, short \textit{MPSoC}s.  This
technique is based on previous work presented in~\cite{Goens}. The
contributions of this thesis are threefold: it elaborates on the theoretical
foundations underlying~\cite{Goens} in greater detail, it describes some
further algorithmic improvements and it presents experimental results obtained
by use of \texttt{mpsym}~\cite{mpsym}, a C++ program written from the ground up
specifically to address symmetry reduction.

We will make use of well established, as well as some recent state of the art
methods originated by researchers in the field of \textit{computational group
theory}, a branch of computational mathematics that concerns itself with the
representation and analysis of many of the algebraic structures underlying
group theory. Refer to e.g.~\cite{Holt} for a comprehensive overview.

We begin by informally describing the problem this thesis addresses in
Section~\ref{sec:mot_problem_statement} and elaborate on what we mean by
symmetry reduction in Section~\ref{sec:mot_symmetry_reduction}.
%
Chapter~\ref{chap:theo} then introduces some necessary fundamentals of
computational group theory in an accessible manner, focussing on the
representation of permutation groups and partial permutation
inverse monoids, algebraic structures we will use to capture the symmetries
inherent in MPSoC architectures.
%
In Chapter~\ref{chap:bsgs} we separately discuss an especially important topic
touched on in Chapter~\ref{chap:theo} in greater detail: construction of a so
called \textit{base and strong generating set} for a given permutation group.
%
In Chapter~\ref{chap:ag} we then outline how to describe common MPSoC
architectures mathematically and how to algorithmically transform these
descriptions into the symmetry capturing algebraic structures introduced in
Chapter~\ref{chap:theo}.
%
Equipped with these fundamentals, we then revisit symmetry reduction in
Chapter~\ref{chap:tmor} were we discuss it in a more formal manner and present
several algorithms to address it.
%
Finally, in Chapter~\ref{chap:exp} we present some experimental results
obtained by use of \texttt{mpsym}.
