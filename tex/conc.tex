In this thesis we have demonstrated how we can simplify MPSoC task mapping by
exploiting symmetries inherent in common MPSoC architectures. In doing so we
have explored some fundamental computational group theory concepts, data
structures and algorithms. We have seen how to describe abstract MPSoC
architecture as graphs and how to extract a representation of their (partial)
symmetries from this description. We have also introduced and formalized the
TMOR problem and devised several algorithms to adress it. Experiments on real
world MPSoC architectures have shown that these algorithms can work well in
practice. However, they are as of now not suitable for use with partial
symmetries due to shortcomings in the way we determine and represent partial
automorphism inverse monoids.

As we have seen in Chapter~\ref{chap:exp}, the overhead incurred by symmetry
reduction is potentially very small, making it a sensible preprocessing step for
most other task mapping algorithms. We have also seen that \texttt{mpsym} is
able to outperform GAP when it comes to solving the TMOR problem. As such,
development of \texttt{mpsym} has paid off.

However, due to the high complexity of many of the algorithms underlying
computational group theory, maintaining and extending \texttt{mpsym} is not
possible without significant mathematical domain knowledge, especially when
correctness and efficiency of these algorithms is of high importance.
One alternative approach might thus be compiling GAP code to C in the spirit of
Cython~\cite{cython} and so directly exposing its interface to a fast compiled
language such as C++.

As far as the current state of \texttt{mpsym} goes, it is mostly complete in
regard to symmetries but lacking in regard to partial symmetries.

For the former, it might be desireable to further improve BSGS construction via
careful analysis of the success rate of the Random Schreier-Sims Algorithm as
well as implementations of the Todd-Coxeter Schreier-Sims and Sims' ``Verify''
Algorithm mentioned in
Section~\ref{sec:bsgs_the_random_schreier_sims_algorithm}. In solving the TMOR
problem, it could furthermore be fruitful to investigate advanced local search
methods as outlined in Section~\ref{sec:tmor_algorithmic_approaches} and to
speed up orbit enumeration beyond the simple fixed point algorithm currently
employed by \texttt{mpym}.

For the latter, a first important step would be faster discovery of a
generating set for the partial automorphism inverse monoid of an architecture
graph, improving over the relatively naive search tree pruning algorithm
presented in Section~\ref{sec:ag_determining_partial_automorphisms} which does
not scale to architectures with even a moderately large number of processing
elements.  Additionally, for deterministically solving the TMOR problem we
would also require a fast way of iterating through an inverse monoid or an
altogether different approach whose complexity remains manageable even for
larger sets of tasks.
