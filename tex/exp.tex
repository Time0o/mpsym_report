This chapter describes experimental results collected during the evaluation of
the most important algorithms presented in previous chapters.  We implement
these algorithms both in the computational algebra system/scripting language
GAP~\cite{GAP4} and in \texttt{mpsym}. All experiments were performed on a
single machine with two Intel\textregistered\ Core\texttrademark\ i5-6200U CPU
cores and 4 GiB DDR3-SDRAM.

Section~\ref{sec:exp_mpsoc_architectures} first describes the abstract MPSoC
architectures for which we perform our experiments. In
Section\ref{sec:exp_automorphism_group_construction} we then examine how fast
we can construct a BSGS for an architectures automorphism group. Finally, in
Section\ref{sec:exp_repr} we compare the performance of the different TMOR
algorithms outlined in Section\ref{sec:tmor_algorithmic_approaches} for large
sets of randomly generated $k$-task mappings, for different values of $k$.

Again, we do not consider partial automorphisms for practical reasons as
indicated in the introduction of Chapter~\ref{chap:tmor}.

\section{Architectures}
\label{sec:exp_mpsoc_architectures}

We perform our experiements for four examplary abstract MPSoC architectures
based on real world MPSoCs. We briefly list and visualize these architectures
here but do not explicitly provide architectures graphs for brevity's sake.  In
the following sections we denote the architecture graph corresponding to an
architecture $X$ by $A_X$ and the automorphism group of $A_X$ by $G_X$.

\begin{figure}
  \centering
    \includeressource[width=.4\textwidth]{exynos.pdf}
  \caption{Exynos architecture with two different types of processing elements.
           Each processing element has its own L1 cache and all processing
           elements of the same type share a separate L2 cache. All processing
           elements can also communicate via shared RAM.}
  \label{fig:exynos}
\end{figure}

\begin{figure}
  \centering
    \includeressource[width=.45\textwidth]{parallella.pdf}
  \caption{Parallelas Epiphany coprocessor architecture with 64 processing
           elements connected in a regular mesh fashion.}
  \label{fig:parallella}
\end{figure}

\begin{figure}
  \centering
    \includeressource[width=.6\textwidth]{haec.pdf}
  \caption{HAEC architecture with four separate SoCs, each similar to
           Parallellas Epiphany coprocessor. The processing elements on one
           SoC can communicate with each other via optical links and with
           \textit{all} processing elements on an adjacent SoC via a wireless
           link.}
  \label{fig:haec}
\end{figure}

\begin{figure}
  \centering
    \includeressource[width=\textwidth]{kalray.pdf}
  \caption{Kalray MMPA-256 architecture made up of 16 identical clusters of 16
           processing elements fully connected via shared memory. The clusters
           themselves are connected in a 2D torus topology.}
  \label{fig:kalray}
\end{figure}

\begin{itemize}
\item The
\textit{Exynos}\footnote{\hyperlink{https://www.samsung.com/semiconductor/minisite/exynos/}{https://www.samsung.com/semiconductor/minisite/exynos/}}
MPSoC developed by Samsung, with \textit{Octa Big-Little} multi core
configuration. See Figure~\ref{fig:exynos}.

\item A 16 by 16 processing element regular mesh based on the
\textit{Epiphany} \cite{Olofsson16} coprocessor of the
\textit{Parallella}\footnote{\hyperlink{https://www.parallella.org/}{https://www.parallella.org/}}
board. See Figure~\ref{fig:parallella}.

\item The \textit{HAEC}~\cite{HAEC} architecture, consisting of several optical
link based MPSoCs connected to each other by wireless links. See
Figure~\ref{fig:haec}.

\item The \textit{Kalray
MMPA-256}\footnote{\hyperlink{https://www.kalrayinc.com/}{https://www.kalrayinc.com/}}
architecture made up of 16 identical compute clusters. See
Figure~\ref{fig:kalray}.
\end{itemize}
%
For reference we also list the orders of these four architectures' automorphism
groups here:
%
\begin{itemize}
\item $|G_{\mathrm{Exynos}}| = 576$.

\item $|G_{\mathrm{Parallella}}| = 8$.

\item $|G_{\mathrm{HAEC}}| = 8192$ and $G_{\mathrm{HAEC}} =
G_{\mathrm{HAEC,proto}} \wr G_{\mathrm{HAEC,super}}$ with
$|G_{\mathrm{HAEC,proto}}| = 8$ and $|G_{\mathrm{HAEC,super}}| = 2$

\item $|G_{\mathrm{Kalray}}| \approx 1.079 \cdot 10^{214}$ and
$G_{\mathrm{Kalray}} = G_{\mathrm{Kalray,proto}} \wr G_{\mathrm{Kalray,super}}$
with $|G_{\mathrm{Kalray,proto}}| = 16!$ and $|G_{\mathrm{Kalray,super}}| = 8$
\end{itemize}

\section{Automorphism Group Construction}
\label{sec:exp_automorphism_group_construction}

\subsection{Methodology}

We first analyse how quickly we can construct a BSGS for a given architectures
automorphism group. This is a prerequisite to solving the TMOR
problem and we will from now on use the term \textit{setup time} to refer to
the overhead incurred by this. Constructing a BSGS for an architecture
graph $A_X$ requires us to perform the following steps:
%
\begin{itemize}
\item 1. Construct a graph data structure representing $A_X$.
\item 2. From this data structure, find a generating set for $G_X$.
\item 3. From this generating set, construct a BSGS for $G_X$.
\end{itemize}
%
For steps 1 and 2, our GAP implementation makes use of the
\textit{grape}~\cite{grape} package and \texttt{mpsym} makes use of the
\textit{Boost Graph Library}~\cite{BGL} and
\textit{nauty}~\cite{nauty}\footnote{grape also uses nauty internally.}, a
program which is able to efficiently determine generating sets for the
automorphism groups of vertex colored graphs\footnote{Remember that we showed
in Section~\ref{sec:ag_determining_automorphisms} how to convert any totally
colored undirected graph to an isomorphic vertex colored one.}

Step 3 is where the Schreier-Sims Algorithm comes into play. GAP does not
directly expose its internal implementations of the different BSGS construction
algorithms. Instead, GAP automatically chooses which algorithm variant or
variants to apply for any given generating set based on complex heuristics.
\texttt{mpsym} currently only implements the ``basic'' Deterministic
Schreier-Sims Algorithm and the Random Schreier-Sims Algorithm.  Since step 3
is usually significantly more computationally expensive than steps 1 and 2 we
only present total execution time data.  We compare the following algorithm
variants:

\begin{itemize}
\item BSGS construction using GAP.

\item BSGS construction using \texttt{mpsym}. We employ both the Deterministic
Schreier-Sims Algorithm on its own and the Random followed by the Deterministic
Schreier-Sims Algorithm (to guarantee correctness).

We also emply the Random Schreier-Sims Algorithm on its own. Note however, that
we do this with the sole purpose of demonstrating how the execution times of a
single run of the Deterministic and Random Schreier-Sims Algorithms compare.
%
We do not attempt to draw a direct comparison between the two algorithms
because the Random Schreier-Sims Algorithm it is not guaranteed to produce a
correct BSGS.
%
Of course we could simply choose $w$ to be suitably large as to make
correctness of the constructed BSGS very likely.
%
But because \texttt{mpsym} currently does not implement any of the algorithms
outlined in Section~\ref{sec:bsgs_the_random_schreier_sims_algorithm} for
detecting/guaranteeing correctness of a BSGS, we have no simple way of checking
whether a BSGS constructed this way is correct. Thus we cannot empirically
verify the relationship between $w$ and the likelihood of a correctly
constructed BSGS\footnote{Unless we verify correctness via complete group
enumeration which is obviously impractical for large automorphism groups.} for
any given generating set. For this reason, we omit this information altogether.
\end{itemize}

\subsection{Results}

\begin{figure}
  \centering
  \inputressource{gap_boxplot.tex}
  \inputressource{mpsym_boxplot.tex}
  \caption{BSGS construction execution times for the Exynos, HAEC and
           Parallella architectures. The value $w = 10$ was chosen for
           \texttt{mpsym}s implementation of the Random Schreier-Sims Algorithm.
           Boxplots were generated from 1000 runs each.}
  \label{fig:bsgs_boxplot}
\end{figure}

Figure~\ref{fig:bsgs_boxplot} visualizes the BSGS construction execution times
for three of our four architectures using both our GAP and \texttt{mpsym}
implementations.  We can make the following observations:

\begin{itemize}
\item For these three architectures with relatively low automorphism group order,
\texttt{mpsym} outperforms GAP significantly.

\item For the HAEC architecture, the difference in execution time between the
Random and Deterministic Schreier-Sims Algorithm, as implemented in
\texttt{mpsym}, is most pronounced.  Nevertheless, combining both algorithms
does not result in a speedup over the deterministic algorithm alone.
\end{itemize}
%
We have purposefully omitted the Kalray architecture from
Figure~\ref{fig:bsgs_boxplot} because deterministically constructing a BSGS for
$G_{\mathrm{Kalray}}$ is computationally very expensive. While GAP is able to
so after several seconds, \texttt{mpsym} is unable to even after several
minutes of execution time. However, we can potentially do much better in two ways:

\begin{itemize}
\item (1) \textit{Avoiding BSGS construction}: Using
Algorithm~\ref{alg:task_repr_wreath_prod} we can solve the TMOR problem without
the need for a BSGS for $G_{\mathrm{Kalray,proto}} \wr
G_{\mathrm{Kalray,super}}$. Instead, we construct BSGSs for
$\sigma(G_{\mathrm{Kalray,super}})$ and
$\sigma(G_{\mathrm{Kalray,proto}})$\footnote{This obviously also applies to
other hierarchical architectures, e.g. HAEC. But since BSGS construction for
$G_{\mathrm{HAEC}}$ is already very fast we only present a comparison for the
Kalray architecture.}.
%
\item (2) \textit{Symmetric group detection}: There exists an efficient and
reliable Monte-Carlo algorithm for testing whether a given generating set
generates a symmetric group, refer to e.g.~\cite{Holt} Chapter 3, such that it
is possible to explicitly construct a BSGS for $S_{16}$ without the need to run
the Schreier-Sims algorithm, even when we are initially unaware that
$G_{\mathrm{Kalray,proto}}$ is a symmetric group.
\end{itemize}

\noindent
Table~\ref{tab:bsgs_kalray} compares the total setup time when using GAP versus
when using \texttt{mpsym} in combination with the methods described above.
Clearly, using method (1) is much less computationally expensive than
constructing a BSGS for $G_{\mathrm{Kalray}}$. Of course taking this approach
implies that we must subsequently use Algorithm~\ref{alg:task_repr_wreath_prod}
to solve the TMOR problem. In Section~\ref{sec:exp_repr} we will analyse
whether this results in further speedup. Using method (2) results in an
additional speedup of about 25\%.

\begin{table}[h!]
  \centering
  \begin{tabular}[t]{lc}
    \toprule
    Method                    & Execution Time \\
    \hline
    GAP                       & $15.31 \pm 0.05s$ \\
    \texttt{mpsym}, (1)       & $3.31 \cdot 10^{-1} \pm 0.01 \cdot 10^{-1}$ \\
    \texttt{mpsym}, (1) + (2) & $2.43 \cdot 10^{-1} \pm 0.07 \cdot 10^{-1}$ \\
    \bottomrule
  \end{tabular}
  \caption{BSGS construction execution times for the Kalray board.
           \texttt{mpsym} makes use of the Deterministic Schreier-Sims Algorithm.
           Means and standard deviations obtained from 10 runs each and
           rounded to nearest two/three decimal places.}
  \label{tab:bsgs_kalray}
\end{table}

\section{Solving the TMOR problem}
\label{sec:exp_repr}

\subsection{Methodology}
\label{sec:exp_repr_methodology}

We now analyse how quickly we are able to solve the TMOR problem for each of
our four architectures.  To this end we generate 10,000 suitable random
$k$-task mappings by uniformly sampling from $\Theta_A^k$ from
Example~\ref{exmp:task_mapping_space} (for $k \in \{4,8,12,16\}$) per
architecture and then determine how long it takes to find orbit representatives
for all of them using each of the following algorithms:

\begin{itemize}
\item Algorithm~\ref{alg:task_repr_iterate}, i.e. bruteforce iteration,
implemented in both GAP and \texttt{mpsym}.

\item Algorithm~\ref{alg:task_repr_orbit} , i.e. bruteforce orbit enumeration,
implemented in GAP. We do not consider an \texttt{mpsym} implementation of this
algorithm here because orbit enumeration as implemented in \texttt{mpsym} is
currently very slow. A possible explanation for this performance discrepancy
between GAP and \texttt{mpsym} is that our GAP implementation makes use of the
highly optimized \textit{orb}~\cite{orb} package.

\item Algorithm~\ref{alg:task_repr_local_search}, i.e. local search,
implemented in \texttt{mpsym}. Since the taks mapping representatives
determined via local search are not guaranteed to be correct, we also note the
success rate of this algorithm, i.e. the ratio of correctly determined orbit
representatives to the number of task mappings.  For the sake of brevity we do
not experiment with any of the possible augmentations to this algorithm
discussed in Section\ref{sec:tmor_algorithmic_approaches}.

\item Algorithm~\ref{alg:task_repr_wreath_prod} , i.e. hierarchical
decomposition, implemented in \texttt{mpsym}. We perform decomposition
explicitly for the HAEC and Kalray architectures. We use this algorithm in
combination with both bruteforce iteration and local search and note any
execution time and/or accuracy improvements.
\end{itemize}

\subsection{Results}
\label{sec:exp_repr_results}

We now present numerical results and analyse if they are consistent with the
considerations we made in
Section\ref{sec:tmor_complexity_and_accuracy_considerations}.  For all figures
in this section, presented execution times are means of ten independent runs
and local search results are annotated with the achieved accuracy.

\begin{figure}
  \centering
  \inputressource{exynos_lineplot.tex}
  \caption{TMOR results for the Exynos architecture.}
  \label{fig:exynos_lineplot}
  %
  \vspace{1cm}
  %
  \inputressource{parallella_lineplot.tex}
  \caption{TMOR results for the Parallella architecture.}
  \label{fig:parallella_lineplot}
\end{figure}

\subsubsection{Exynos}

Starting with the results for the Exynos architecture presented in
Figure~\ref{fig:exynos_lineplot} we can already make several important
observations:
%
\begin{itemize}
  \item Bruteforce orbit enumeration is faster than bruteforce iteration for
        the chosen values of $k$ but its performance, as expected, depends
        more strongly on $k$.
  \item The \texttt{mpsym} implementation of bruteforce iteration is more than an
        order of magnitude faster than the corresponding GAP implementation.
  \item Local search is significantly faster than bruteforce iteration and
        returns the correct representative for all 10,000 task mappings.
\end{itemize}

\subsubsection{Parallella}

The results for the Parallella architecture presented in
Figure~\ref{fig:parallella_lineplot} exhibit similar trends. Here however,
bruteforce iteration is faster than orbit enumeration for the chosen
values of $k$ which is most likely due to the very small order of Parallellas
automorphism group. Nevertheless, local search is still faster than bruteforce
iteration and achieves perfect accuracy as well.

\subsubsection{HAEC}

\begin{figure}
  \centering
  \inputressource{haec_lineplot.tex}
  \caption{TMOR results for the HAEC architecture. GAP data is extrapolated
           from execution times measured on just 100 random task mappings.
           Data series marked with ``*'' do \textbf{not} make use of
           hierarchical decomposition.}
  \label{fig:haec_lineplot}
  %
  \vspace{1cm}
  %
  \inputressource{haec_histogram.tex}
  \caption{Orbit size distribution for 100 random task mappings for the HAEC
           architecture, for different values of $k$. $n$ refers to the number
           of occurences of orbits of the corresponding size.}
  \label{fig:haec_histogram}
\end{figure}

The HAEC architecture, for which results are presented in
Figure~\ref{fig:haec_lineplot} is the first for which the performance of
bruteforce iteration and orbit enumeration diverge drastically. While
$|G_{\mathrm{HAEC}}|$ is relatively small, the average
size of its orbits grows quickly with $k$. This is further demonstrated in
Figure~\ref{fig:haec_histogram} which visualizes how the size of more and more
orbits approaches or equals $|G_{\mathrm{HAEC}}|$ as $k$ grows\footnote{Note that
all orbit sizes divide $|G_{\mathrm{HAEC}}|$.}. Furthermore we can observe that
hierarchically decomposing HAECs automorphism group results in a speedup of
almost three orders of magnitude when used in combination with bruteforce
iteration and a jump from mediocre to perfect accuracy when used in combination
with local search.

\subsubsection{Kalray}

\begin{figure}
  \inputressource{kalray_lineplot.tex}
  \caption{Results for the Kalray architecture. Both data series make use
           of hierarchical decomposition and transitive group optimization.}
  \label{fig:kalray_lineplot}
\end{figure}

The benefits of hierarchical decomposition are made even more clear by the
results for the Kalray architecture presented in
Figure~\ref{fig:kalray_lineplot}.  Here, without architecture graph
decomposition, bruteforce iteration, implemented in both GAP and
\texttt{mpsym}, as well as bruteforce orbit enumeration are not able to find a
representative for \textbf{a single} task mapping even after executing for
several minutes and local search is not able to accurately determine even a
single representative.  Finding representatives for thousands of task mappings
without hierarchical decomposition is thus completely infeasible.

With architecture graph decomposition alone, the situation still looks dire:
Each of the clusters making up the Kalray architecture has the automorphism
group $G_{\mathrm{Kalray,proto}} = S_{16}$ and similarly, all
$\sigma_i(G_{\mathrm{Kalray,proto}}) \in \sigma(G_{\mathrm{Kalray,proto}})$ are
isomorphic to $S_{16}$.  Because $|S_{16}| = 16!$, albeit much smaller than the
complete automorphism group, is still very large, bruteforce iteration remains
slow and local search remains unacceptably inaccurate. This is however easily
remedied because symmetric permutation groups are transitive which allows us to
trivially solve the TMOR problem for all $\sigma_i(G_{\mathrm{Kalray,proto}})
\in \sigma(G_{\mathrm{Kalray,proto}})$ (since there is only one orbit, the
representative is the same for every possible task mapping). Since in this case
$|G_{\mathrm{Kalray,super}}|$ is small, the overall TMOR problem becomes
feasible.

Figure~\ref{fig:kalray_lineplot} visualizes results obtained using both
architecture graph decomposition and the aforementioned transitive group
optimization. Finding correct representatives for all 10,000 task mappings
takes less than a second in total using both bruteforce iteration and local
search and the latter achieves perfect accuracy as well.
